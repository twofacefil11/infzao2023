\documentclass{article}
\usepackage{unicode-math}
\usepackage[margin=2cm]{geometry}
\usepackage{fontspec}
\usepackage{xcolor}
\usepackage{unicode-math}
\usepackage{pagecolor}
\usepackage{amsmath}

\setlength{\parindent}{0pt}
\title{Algera Boola}
\setmainfont{Consolas} 
\pagecolor{black}      
\color{white}          

\begin{document}
\begin{titlepage}
    \begin{center}
        \vspace*{1cm}
        
        \Huge

        \textbf{PODSTAWY INFORMATYKI}
        
        \vspace{3cm}
    \end{center}
\end{titlepage}
\newpage
\section{PRAWA RACHUNKU ALGEBTY BOOLA}
\subsection{Idempotentoność}
\[a + a = a\]
\[a \cdot a = a\]
\subsection{przemienność}
\[a + b = b + a\]
\[a \cdot b = b \cdot a\]
\subsection{Łączność}
\[(a + b) + c = a + (b + c)\]
\[(a \cdot b) \cdot c = (a \cdot b) + c = (a + c) \cdot (b + c)\]
\subsection{absorbcja}
\[a + (a + b) - a\]




blah blah what now ?
% Rest of your document goes here

\end{document}




