\documentclass{article}
\usepackage{unicode-math}
\usepackage[margin=3cm]{geometry}
\usepackage{fontspec}
\usepackage{xcolor}
\usepackage{unicode-math}
\usepackage{pagecolor}
\usepackage{amsmath}


\setlength{\parindent}{0pt}
\title{Algera Boola}
\setmainfont{Consolas} 
\pagecolor{black}      
\color{white}          

\begin{document}
\begin{titlepage}
    \begin{center}
        \vspace*{1cm}
        
        \Huge

        \textbf{PODSTAWY INFORMATYKI}
        
        \vspace{3cm}
    \end{center}
\end{titlepage}
\newpage
\section{PRAWA RACHUNKU ALGEBTY BOOLA}
\subsection{Idempotentoność}
\[a + a = a\]
\[a \cdot a = a\]
\subsection{przemienność}
\[a + b = b + a\]
\[a \cdot b = b \cdot a\]
\subsection{Łączność}
\[(a + b) + c = a + (b + c)\]
\[(a \cdot b) \cdot c = (a \cdot b) + c = (a + c) \cdot (b + c)\]
\subsection{Rozdzielność}
\[(a + b) \cdot c = a \cdot c + b \cdot c\]
\[(a \cdot b) \cdot c = (a \cdot b) + c = (a + c) \cdot (b + c)\]
\subsection{absorbcja}
\[a + (a + b) = a\]
\[a * a ????\]
\subsection{pochłanianie}
\[a + a = 1\]
\[a \cdot a = 0\]
\subsection{'0' jest elementem neutralnym dla dodawania}
\[a + 0 = a\]
\[a \cdot 0 = 0\]
\subsection{'1' jest elementem neutralnym dla mnożenia}
\[a + 1 = 1\]
\[a \cdot 1 = a\]
\begin{center}
    \textit{'Najprostsza nietrywialna algebra Boola jes dwuelementowa, więc $B = {0,1}$'}
\end{center}
\subsection{Dodawanie}
\[
\begin{array}{c|c|c}
    + & 0 & 1 \\
    \hline
    0 & 0 & 0 \\
    \hline
    1 & 1 & 1 \\
\end{array}
\]

\subsection{Mnożenie}
\[
    \begin{array}{c|c|c}
        \cdot & 0 & 1 \\
        \hline
        0 & 0 & 0 \\
        \hline
        1 & 0 & 1 \\
    \end{array}
\]

\section{Tabele Prawdy}
Czyli ładniej rospisane to co powyżej.
\subsection{Dodawanie}
\[
\begin{array}{c|c}
    a b & a + b \\
    \hline
    00 & 0 \\
    01 & 1 \\
    10 & 1 \\
    11 & 1 \\
\end{array}    
\]
\subsection{Mnożenie}
\[
\begin{array}{c|c}
    a b & a + b \\
    \hline
    00 & 0 \\
    01 & 1 \\
    10 & 1 \\
    11 & 1 \\
\end{array}    
\]
\end{document}




